\section{Introduction}

Retinomorphic event cameras are biologically inspired cameras that unlike traditional cameras do not capture entire intensity of the visual space at a given time. Instead, these sensors capture any change in intensity of pixels in the visual space (beyond a certain threshold) without being bounded by time \cite{boahen2005neuromorphic}. This asynchronous nature and sensitivity to scene dynamics have led to a lot of advantages namely low latency, high temporal resolution, high dynamic range and low power consumption compared to conventional frame based camera.\\

The first retinomorphic camera was developed by Mahowald and Mead. DVS and consecutively DAVIS are the two currently used event camera desgins.  
 My lab rotation deals with implementing an image reconstruction algorithm and comparing it with a CNN algorithm ; further I intend to try a different dictionary learning algorithm to represent the events. The deliverable of the lab rotation is enumerated below :-

\begin{itemize}
  \item History of Image reconstruction in Event camera.
  \item Study and implementation of Image Reconstruction by dictionary learning. \cite{barua2016direct}
  \item Study, implementation and analysis of FireNet.\cite{scheerlinck2020fast}
  \item Comparative analysis of the two algorithms.
\end{itemize}